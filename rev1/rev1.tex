\documentclass[conference]{IEEEtran}
\usepackage[utf8]{inputenc}
\usepackage[T1]{fontenc}

% Graphics
\usepackage{graphicx, float, subfigure, blindtext}
\newcommand\IEEEhyperrefsetup{
    bookmarks=true,bookmarksnumbered=true,%
    colorlinks=true,linkcolor={red},citecolor={red},urlcolor={black}%
}

% Preferred hyperref setup, Michael Shell
\usepackage[\IEEEhyperrefsetup, pdftex]{hyperref}

% These packages must be at the end
\usepackage[nolist,nohyperlinks]{acronym}
\usepackage{cleveref}
\graphicspath{{images/}}

% Remove section first paragraph indent
\usepackage{titlesec}
\titlespacing*{\section}{0pt}{*1}{*1}
\titlespacing*{\subsection}{0pt}{*1}{*1}
\renewcommand{\thesubsubsection}{\arabic{subsubsection}}
\titleformat{\subsubsection}[runin]{\itshape}{\thesubsubsection)}{1em}{}[:]
\titlespacing*{\subsubsection}{\parindent}{0pt}{*1}


% The article title.
\title{Peer Review for \textit{How big companies benefit from Lean Startup}}

% Document begins here
\begin{document}

    \author{}

% Create the title.
    \maketitle

    While this article does conform to formatting requirements, it includes some unfortunate
    typos~(such as \textit{`approxiamtely'}), grammar mistakes~(like forgetting determinants -
    \textit{`product'} vs. \textit{`a product'}), typography missteps~(like forgetting
    whitespace before a parenthesis), as well as conjugation errors (\textit{`Drew Houston [...],
    was struggled...')}.


    \section{Introduction}

    Citing specific principles of lean startup methodology is great - it also
    allows the reader to look those up on their own.


    \section{Body}
    The lack of paragraphs throughout the entire section makes reading it
    slightly difficult.

    \subsection{MVP Subsection}


    While the section is well-documented and factually correct, it goes to great lenghts to explain
    some lean ideas (like what an \emph{MVP} is at all) and list their benefits
    without actually making much of a point in how these benefit specifically large
    companies.

    The examples given, while also correct, are actually examples of startups
    (Dropbox, Amazon...) which makes it even harder to see how lean
    ideas are applicable \emph{to companies other than} startups.

    \subsection{Innovation Accounting subsection}

    The article does assume the only driver for large companies' revenue is
    innovation~(`\textit{Innovation drives the sustainable development of these large companies}')
    - but one could argue Facebook's main revenue stream is its advertising business
    (rather than its VR and AR spinoffs) and that Apple's is smartphones (and not music and video
    streaming).

    Like the article itself argues, some companies' focus can be maintenance instead
    of innovation.
    How does lean methodology help there? \\

    \section{Review Veredict}

    While well cited, factual, and somewhat structured (it does well to limit its scope the article
    only somewhat answers the given question.

    It goes on to mainly list
    benefits of lean methodology rather than pointing out how they might be useful outside of the
    innovation business of startups: not all companies are looking to disrupt markets!
    The article answers to `\emph{How big companies benefit from Lean Startup}' rather than
    `\emph{To what degree can larger corporations benefit from lean startup ideas?}'.

    The difference is subtle but important: the latter also encourages to question
    whether lean startup ideas might hinder large corporations and actually have drawbacks.
    This article mostly says `Lean ideas are good because ...' but fails to explain
    where they should not be applied.
    Only a brief consideration is made in this direction with the
    last sentence (`\textit{However, it’s worth noting that Lean Startup is not suitable for mature
    businesses within large companies, where the focus is no longer innovation but maintenance.}'),
    which is great, but the article failed to go beyond the obvious and say \emph{why} this is the
    case, or to what extent.\\

    On the other hand, the level of presumed knowledge is great!
    The article goes through the trouble of explaining the Build-Measure-Learn and Innovation
    Accounting principles, as well as `the point' of lean methodology.
\end{document}
